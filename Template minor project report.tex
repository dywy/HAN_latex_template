%% HAN Latex Project Report
%% Preamble starts here
\documentclass[11pt,a4paper,twoside]{article}
\usepackage[utf8]{inputenc}

%% Here's where the title and author are defined.
\usepackage{authoraftertitle}
\title{Template to create a minor project report}
\author{My Name}
\date{\today}
%\institution{HAN University of Applied Sciences}


%% Setting up document style
\usepackage{enumitem}
\usepackage[left=2cm,right=2cm,top=2cm,bottom=2cm]{geometry}
\setlength{\parindent}{0mm}
\usepackage{fancyhdr}
\setlength{\headheight}{15.2pt}
\pagestyle{fancy}
\fancyhf{}
\fancyhf[FC]{\thepage}
\fancyhf[FLO,FRE]{\MyTitle}


%% Stuff for external references.
\usepackage[backend=biber,natbib=true]{biblatex} 
\addbibresource{../references.bib}
\usepackage{csquotes}


%% For placing code, equations and units
\usepackage{listings}
\usepackage[framed]{matlab-prettifier}
\usepackage{sistyle}
\usepackage{mathrsfs}
\usepackage{amsmath, amsfonts,amssymb}
\usepackage{hyperref}


%% For Images, graphics et al
\usepackage{import}
\usepackage{graphicx}
\usepackage{grffile}
\usepackage{tikz}
\usetikzlibrary{calc}
\usepackage{transparent}
\usepackage{rotating}

%% Temp stuff
\usepackage{blindtext}
\usepackage[absolute]{textpos}

%% End of preamble. Document begins here.
\begin{document}

%% Create a title page with the logo, the necesary info and no headers or footers
\begin{titlepage}
\thispagestyle{empty}

\begin{picture}(0,0)
\put(250,-60){\hbox{\includegraphics[scale=2.5]{images/HAN_logo.png}}}
\end{picture}


\begin{textblock}{8}(6 ,4)
\Large
\hspace*{\fill} \textbf{HAN Master Project}
\vspace{0.5cm}
\begin{flushright}
\Huge
\textbf{\MyTitle}
\end{flushright}
\end{textblock}

\begin{textblock}{8}(6 ,12)
\begin{flushright}
\Large
\textbf{Arnhem, \today \\ \MyAuthor}
\end{flushright}
\end{textblock}

\end{titlepage}



%% Set pagenumbering to roman
\pagenumbering{roman}

%% Page with preface
\section*{Preface}
Your Preface text here.


%% Page with summary
\newpage
\section*{Summary}
Your summary here.


%% Page with table of contents
\newpage
\tableofcontents

\newpage
\pagenumbering{arabic}

\section{Introduction}
Awesome introduction.


\section{First Section}


\subsection{Equations}
Here's an example of an equation
\begin{equation}\label{EQ}
\ddot{x}=
-\frac{c}{m}\dot{x}^2
-\frac{b}{m}\dot{x}
-\frac{F_b}{m}
+\frac{F_a}{m}
\end{equation}

It can be referred to like such: \ref{EQ},

\subsection{Code listing}
Here's how to include a matlab .m file. See listing \ref{matlabfile}.

\lstinputlisting[label=matlabfile,
				style=Matlab-editor,
				caption=MATLABFILE.m]
				{../code/MATLABFILE.m}
				

Here's how to format text into matlab format in-line:
\begin{lstlisting}[language=matlab]
% MATLABFILE.m
n = input('Enter a number:');
for sentence = 1:n		% have an indexing go from 1 to n
    fprintf('%d. Hello world!\n', sentence) % print the index, and a newline \n end
\end{lstlisting}


\subsection{Tables}
Here's a table:

\begin{table}[h]
  \centering
    \begin{tabular}{| l l l |}
    \hline
    Column 1 & Column 2 & Column 3 \\
    \hline
    4 & 5 & 6 \\
    7 & 8 & 9 \\
    \hline
    \end{tabular}
  \caption{A simple table}
\end{table}

Consider checking out \url{https://www.tablesgenerator.com/}.

\subsection{Referencing}
Make a reference to \citep{NormanS.Nise2012}.

\section{Conclusion}
Conclusion

\newpage
\section*{References}
\printbibliography[title={\null}]

\newpage
\section*{Appendix A Nomenclature}

\end{document}
